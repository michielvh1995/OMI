\documentclass[a4paper,12pt]{article}
\usepackage[utf8]{inputenc}

\title{Research Plan Force-directed Graph Drawing}
\author{Michiel van Heusden, 4173309 \and Maurits van der Veen, 4167287 \and Kevin Oosterlaak, 4012372}
\date{\today}

\begin{document}
  \maketitle
  \tableofcontents
  \section{intro}
  Data requires often visualization before people understand what it means.
  This visualization is done with graphs and charts.
  This research focusses on the drawing methods of graphs.
  These graphs consist of objects and the relations between eachother.
  In graph theory the objects are called vertices and their relations edges.
  % Source
  % Bondy, John A.; Murty, Uppaluri S. R. (1976), “Graph theory with applications”, The Macmillan Press Ltd. Vol. 290.
  Generating readable graphs becomes difficult according to the amount of vertices and edges.
  To generate these correctly many algorithms were developed.
  One family of functions to do this is called \emph{Force Directed Graph Drawing}.
  The idea behind force directed graph drawing is to use physics based algorithms to calculate the positions of each correctly.
  In this research a few algorithms wil be examined and how correctly they can generate a graph.
  Not only will we compare different algorithms, we will also inspect how some algorithm-specific constants influence the results.

  The algorithms that will be subject to our research are as follows:
  \begin{itemize}
    \item Hooke-Coulomb's Algorithm
    \item The Fruchterman Reingold Algorithm
    \item Eades' algorithm
  \end{itemize}
  More on these and the constants are in chapter \ref{par:algorithms}.

  \section{Algorithms}\label{par:algorithms}

  \section{Research Question}
    Which force directed graph drawing algorithm yields the highest quality graphs?

  \section{Hypothesis}

  \section{Problem Discription}


  \section{Scope and Assumptions}
  \section{Criteria}
  \section{Test Data}

  \section{Scenarios}

\end{document}
